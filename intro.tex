%%%%%%%%%%%%%%%%%%%%%%%%%%%%%%%%%%%%%%%%%%%%%%%%%%%%%%%%%%%%%%%%%%%%%%%%%%%%%%%
\section{Introduction}
\label{sec:intro}
%%%%%%%%%%%%%%%%%%%%%%%%%%%%%%%%%%%%%%%%%%%%%%%%%%%%%%%%%%%%%%%%%%%%%%%%%%%%%%%

%=============================================================================
\subsection{The MoEDAL experiment}
\label{sec:moedalintro}
%=============================================================================

The Monopole and Exotics Detector at the LHC~\cite{MoEDAL2009}
is the latest addition to a long line of experiments that have searched for
Dirac's hypothesised magnetic monopole.
%
Based at the Large Hadron Collider's Interaction Point 8 (IP8),
it is housed in the same experimental cavern as the LHCb experiment.
%
The three major subdetectors -- the Nuclear Track Detectors (NTDs),
Magnetic Monopole Trapping (MMT) detectors, and
Timepix detector array --
are largely situated in and around the LHCb VeLo (Vertex Locator) housing.
%
It is hoped that these subdetectors will provide experimental evidence
for the production of magnetic monopoles or other highly-ionising
Beyond Standard Model (BSM) physics produced in the LHC's
highest energy laboratory-based
proton-proton, proton-lead, or lead-lead collisions as -- to date --
no such evidence has been reported in any of publications featured here.

%=============================================================================
\subsection{The scope of this document}
\label{sec:scope}
%=============================================================================
This document aims to provide a bibliography for all those
with an interest in the work of the MoEDAL Collaboration -- be they
collaboration members, students associated with MoEDAL-related research,
or members of the public --
to aid with background reading, literature searches, and
preparing publications.
%
It should cover publications relating to the theory behind
and experimental searches for magnetic monopoles and
other highly-ionising phenomena that are within reach
of the MoEDAL experiment's detector systems.
%

While such a document can never be fully comprehensive,
the author(s) will endeavour to make it as complete as possible.
Additions, comments and suggestions can and should be submitted to
the document's GitHub repository\footnote{
\href{https://github.com/CERNatschool/moedal-bibliography}{https://github.com/CERNatschool/moedal-bibliography}
}.

%=============================================================================
\subsection{Using this bibliography}
\label{sec:using}
%=============================================================================
The bibliography is presented in the standard BibTeX \texttt{unsrt} style
in the final section of this document.
However, the \texttt{BibTeX} source file may be found
in this document's GitHub repository
and its contents re-used as necessary.

%=============================================================================
\subsection{Overview}
\label{sec:overview}
%=============================================================================
After this introduction (Section~\ref{sec:intro}),
quick reference summary tables of the publications
(with BibTeX citation codes and brief notes)
are provided in Section~\ref{sec:summaries}.
%
The full bibliography follows.
